\subsubsection{Racine \'enieme}
Soient $w$ et $z$ deux nombre complexes, on cherche a résoudre l'équation
$w^n = z$ avec $n \in \mathbb{N}^*$. Pour $n>2$, on utilisera le plus souvent la forme exponentielle.

\begin{definition}{racine n-ieme de l'unité}
	Resolvons $w^n = 1$
	\begin{enumerate}
		\item on emploie la forme polaire, $w = r(\cos(\theta)+i\sin(\theta))$ 
		\item n est un entier positif, donc le nombre complexe a n racines
		\item la formule pour obtenir un nombre complexe est 
			\begin{align*}
				\sqrt[n]{r}(\cos\alpha+i\sin\alpha) \\
				\alpha = \frac{\theta+2k\pi}{n}
			\end{align*}
 	\end{enumerate}
	on obtient donc, pour $w = 4i$:
	\begin{align*}
		4i = 4(\cos\frac{\pi}{2}+i\sin\frac{\pi}{2})\\
		k=0, alpha = \frac{\frac{\pi}{2}}{2} =  \frac{\pi}{4} \\
			w_1= 2(\cos\frac{\pi}{4}+i\sin\frac{\pi}{4}) \\
			\Rightarrow w_1 = 2(\frac{\sqrt{2}}{2}+i\frac{\sqrt{2}}{2}) = \sqrt{2}+i\sqrt{2}
	\end{align*}
\end{definition}
