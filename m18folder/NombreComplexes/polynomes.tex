\subsection{Les Polynomes}
Un polynome est définit comme suit
\begin{equation*}
	P(x) = 0+0_1X+0_2X^2+0_3X^3+...+0_nX^n 
\end{equation*}
polynome comme objet abstrait P \\
\begin{example}
	$p(x)=x^2$ comme polynome
\end{example}
\begin{example}{En Particulier}
	$P=0$ comme polynome \\
	% je veux dire aue P doit etre
\end{example}

\begin{definition}
	on dit que $z_0$ est une racine de P tel que $P(z_0)=0$ \\
	On utilie le mot racine dans les deux senses
	\begin{itemize}
		\item racine (n-ieme) d'un nombre complexe tel que $z_0^n=w$
		\item racine d'un polynome tel que $P(z_0)=0$
	\end{itemize}
\end{definition}

\begin{definition}
	Soit $P$ un polynôme \\
	on appelle degré en polynôme \\
	la puissance la plus grande pour laquelle $n \neq 0$
	\begin{example}
		\begin{align*}
			P(x) = 1+x^2+x^27\\ 
			deg=27
		\end{align*}
	\end{example}
	\begin{definition}{Question}
		Soit P un polynôme de degré n
		combien de racine admet-t-il, (complexe et réel)?
		\begin{example}
			\begin{equation*}
				P(z)=z^n-1
   			\end{equation*}
			2 racines réels, et 2 racines complexes
			\begin{align*}
				z_0 racines \iff z_0^2 = 0 \\
				donc \\
				z_0 \quad \text{est la racine}
   			\end{align*}
		\end{example}
	\end{definition}
\end{definition}


\begin{definition}{Division des polynomes}
	\begin{theorem}
		Soit $F$ et $G$ des polynômes, ($degF\leq degG$)\\ 
		alors ils existent $Q$ et $R$ tels que $degR<degG$
		\begin{align*}
			F=QG+R
  		\end{align*}
		Si $R=0$, on dit aue F est divisible par G 
		% \begin{definition}{Algorithme de la division euclidienne}
		% \end{definition}
		\begin{example}
			Voici un exemple de division polynomial
			\begin{table}
			\begin{tabular}{|c|c|}
				\hline
   				$x^5+x^5+0x^3+3x^2+x+1$ & $x^3+2x+1$ \\ 	
				\hline 
				$x^5+2x^3+x2$ & $x^2+x-2$ \\
				\hline
   			\end{tabular}
			\end{table}
		\end{example}
	\end{theorem}
	% \begin{theorem}{cv} 
\end{definition}
