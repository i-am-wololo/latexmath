\subsection{Racine carr\'ee}
Soit z un complexe non nul, on cherche a r\'esoudre l'equation $w^2=z$ \\
on peut \'ecrire $w$ et $z$ sous forme cart\'esienne et identifier:\\
si $z=a+ib, (a,b) \in \mathbb{R}^2$ et $w = \alpha + i\beta, (\alpha, \beta) \in \mathbb{R}^2$, on obtient:
% \begin{align*}
% 	\alpha^2+\beta^2=a \quad et \quad 2\alpha\beta=b
% \end{align*}
On obtient aussi, d'apr\`es l'\'egalit\'e des modules, $\alpha^2+\beta^2=\sqrt{a^2+b^2}$
