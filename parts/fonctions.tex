\section{fonctions}
	\subsection{definitions}

	\subsection{Logarithme n\'ep\'erien}
	\textbf{Propri\'et\'es}: \\
	\begin{align*}
		\ln \rightarrow \mathbb{R}^+_*,\\
		\frac{d}{dx}\ln \rightarrow \frac{1}{x}\\
		\ln(1) = 0
	\end{align*}
	\subsubsection{Propri\'et\'es du logarithme}
		\begin{itemize}
			\item $\ln$ est une bijection strict-crois de $\mathbb{R}_+^*$
			\item Pour tout r\'eels $a>0$ et $b>0$:
			\begin{equation*}
   				\ln(ab) = \ln(a)+\ln(b)	
   			\end{equation*}
		\item Pour tout r\'el $a>0$ et tout entier relatif $n \in \mathbb{Z}$:
		\begin{equation*}
			\ln(a^n)=a\ln(a)
    		\end{equation*}
			
		\end{itemize}
	
	\newtheorem{definition}{$\log_a$}
	\begin{definition}
		\begin{equation}
			\log_a:x \rightarrow\log_a(x)=\frac{\ln(x)}{\ln(a)}
  		\end{equation}
	\end{definition}
	\subsubsection{Propri\'et\'es du logarithme en base $a$}
		\begin{itemize}
			\item La fonction $\log_a$ est une bijection de $$\mathbb{R}^*_+$$
			\item on a:
			\begin{equation*}
					\log_a(1)=0\quad et \quad log_a(a)=1	
    			\end{equation*}
			\item pour tout r\'eels $\{c, d\} >0$:
			\begin{align*}
    				log_a(cd)=log_a(c)+log_a(d) \\
				log_a(\frac{c}{d})=log_a(c)-log_a(d)
    			\end{align*}
			\item pour tout r\'eel $c>0$, on a
			\begin{equation*}
   				\log_a(ac)=\log_a(c)+1		
   			\end{equation*}
			\item on suppose que $a>1$. Donc, $\forall c>0$ on a: \\
				$n$ est la partie enti\`ere de $\log_a(c)$ (c\`ad $n \leq \log_a(c) < n+1$) si et seulement si
				\begin{equation*}
					a^n\leq c < a^{n+1}
   				 \end{equation*}	
		\end{itemize}
		\newtheorem{example}{Exemple}
		\begin{example}
			Dans le cas du log en base 10, on obtient:
			\begin{equation*}
				\log_{10}(10c)=\log_10(c)+1
			\end{equation*}
			et la partie enti\`ere de $\log_10(c)$ est le nombre entier $n$ tel que:
			\begin{equation}
				10^n\leq x < 10^{n+1}
			\end{equation}
			ou autrement dit $c \in [10^n,10^{n+1}]$
		\end{example}

	\subsection{exponentielle}

	% \subsection{quelque graphes}
	

